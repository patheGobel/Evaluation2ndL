\documentclass[12pt,a4paper]{article}
\usepackage{amsmath,amssymb,mathrsfs,tikz,times,pifont}
\usepackage{enumitem}
\newcommand\circitem[1]{%
\tikz[baseline=(char.base)]{
\node[circle,draw=gray, fill=red!55,
minimum size=1.2em,inner sep=0] (char) {#1};}}
\newcommand\boxitem[1]{%
\tikz[baseline=(char.base)]{
\node[fill=cyan,
minimum size=1.2em,inner sep=0] (char) {#1};}}
\setlist[enumerate,1]{label=\protect\circitem{\arabic*}}
\setlist[enumerate,2]{label=\protect\boxitem{\alph*}}
%%%::::::by chnini ameur :::::::%%%
\everymath{\displaystyle}
\usepackage[left=1cm,right=1cm,top=1cm,bottom=1.7cm]{geometry}
\usepackage[colorlinks=true, linkcolor=blue, urlcolor=blue, citecolor=blue]{hyperref}
\usepackage{array,multirow}
\usepackage[most]{tcolorbox}
\usepackage{varwidth}
\usepackage{float} %pour utiliser l'option [H] qui force l'image à apparaître exactement à l'endroit où elle est placée dans le code.
\tcbuselibrary{skins,hooks}
\usetikzlibrary{patterns}
%%%::::::by chnini ameur :::::::%%%
\newtcolorbox{exa}[2][]{enhanced,breakable,before skip=2mm,after skip=5mm,
colback=yellow!20!white,colframe=black!20!blue,boxrule=0.5mm,
attach boxed title to top left ={xshift=0.6cm,yshift*=1mm-\tcboxedtitleheight},
fonttitle=\bfseries,
title={#2},#1,
% varwidth boxed title*=-3cm,
boxed title style={frame code={
\path[fill=tcbcolback!30!black]
([yshift=-1mm,xshift=-1mm]frame.north west)
arc[start angle=0,end angle=180,radius=1mm]
([yshift=-1mm,xshift=1mm]frame.north east)
arc[start angle=180,end angle=0,radius=1mm];
\path[left color=tcbcolback!60!black,right color = tcbcolback!60!black,
middle color = tcbcolback!80!black]
([xshift=-2mm]frame.north west) -- ([xshift=2mm]frame.north east)
[rounded corners=1mm]-- ([xshift=1mm,yshift=-1mm]frame.north east)
-- (frame.south east) -- (frame.south west)
-- ([xshift=-1mm,yshift=-1mm]frame.north west)
[sharp corners]-- cycle;
},interior engine=empty,
},interior style={top color=yellow!5}}
%%%%%%%%%%%%%%%%%%%%%%%

\usepackage{fancyhdr}
\usepackage{eso-pic}         % Pour ajouter des éléments en arrière-plan
% Commande pour ajouter du texte en arrière-plan
\usepackage{tkz-tab}
\AddToShipoutPicture{
    \AtTextCenter{%
        \makebox[0pt]{\rotatebox{80}{\textcolor[gray]{0.5}{\fontsize{5cm}{5cm}\selectfont PGB}}}
    }
}
\usepackage{lastpage}
\fancyhf{}
\pagestyle{fancy}
\renewcommand{\footrulewidth}{1pt}
\renewcommand{\headrulewidth}{0pt}
\renewcommand{\footruleskip}{10pt}
\fancyfoot[R]{
\color{blue}\ding{45}\ \textbf{2025}
}
\fancyfoot[L]{
\color{blue}\ding{45}\ \textbf{Prof:M. BA}
}
\cfoot{\bf
\thepage /
\pageref{LastPage}}
\begin{document}
\renewcommand{\arraystretch}{1.5}
\renewcommand{\arrayrulewidth}{1.2pt}
\begin{tikzpicture}[overlay,remember picture]
\node[draw=blue,line width=1.2pt,fill=purple,text=blue,inner sep=3mm,rounded corners,pattern=dots]at ([yshift=-2.5cm]current page.north) {\begingroup\setlength{\fboxsep}{0pt}\colorbox{white}{\begin{tabular}{|*1{>{\centering \arraybackslash}p{0.28\textwidth}} |*2{>{\centering \arraybackslash}p{0.2\textwidth}|} *1{>{\centering \arraybackslash}p{0.19\textwidth}|} }
\hline
\multicolumn{3}{|c|}{$\diamond$$\diamond$$\diamond$\ \textbf{Lycée de Dindéfélo}\ $\diamond$$\diamond$$\diamond$ }& \textbf{A.S. : 2024/2025} \\ \hline
\textbf{Matière: Mathématiques}& \textbf{Niveau : 2nd}\textbf{L} &\textbf{Date: 14/02/2025} & \textbf{Rendre le 19 à 10H} \\ \hline
\multicolumn{4}{|c|}{\parbox[c]{10cm}{\begin{center}
\textbf{{\Large\sffamily Devoir à faire à la maison}}
\end{center}}} \\ \hline
\end{tabular}}\endgroup};
\end{tikzpicture}
\vspace{3cm}

\section*{\underline{Exercice 1 :}}
\begin{enumerate}
    \item Calculer les expressions suivantes et donner les résultats sous la forme de fractions irréductibles :
    \[
    \mathbf{A} = \left( -1 + \frac{1}{\frac{1}{4} + \frac{1}{2}} \right) \times \left( 1 - \frac{1}{2} \right)
    \]
    \[
    \mathbf{B} = \left( 1 - \frac{1}{3} \right) \times \left( 3 - \frac{\frac{3}{2}}{1 + \frac{1}{3}} \right)
    \]
    
    \item Écrire sous la forme \( a\sqrt{b} \) l’expression suivante :
    \[
    2\sqrt{28} + 2\sqrt{63} + 3\sqrt{7}
    \]
    
    \item Écrire l’expression suivante sous la forme \( 2^m \times 5^n \times 7^p \) :
    \[
    A = \frac{7^5 \times 4^2 \times 5^6}{5^3 \times 7^3 \times 8^3}
    \]
\end{enumerate}

\section*{\underline{Exercice 2 :}}

\textit{On donne les trinômes suivants \( A, B, C \) et \( D \)}

\[
A = 5x^2 - 7x - 34 \quad B = 2x^2 - 5x + 3
\]
\[
C = -5x^2 + 9x - 5 \quad D = 2x^2 - 6x + 5
\]

\begin{enumerate}
    \item Mettre les trinômes ci-dessus sous forme canonique 
    \item Factoriser si possible les trinômes \( A, B, C \) et \( D \) 
\end{enumerate}

\bigskip

\section*{\underline{Exercice 3 :}}

Résoudre dans \( \mathbb{R} \) les équations suivantes :

\begin{enumerate}
    \item[(a)] \( 3x^2 - 5x + 11 = 0 \) 
    \item[(b)] \( x^2 - 5x + 6 = 0 \) 
    \item[(c)] \( -4x^2 + 28x - 49 = 0 \) 
\end{enumerate}

\bigskip

\section*{\underline{Exercice 4 :}}

Résoudre dans \( \mathbb{R} \) les inéquations suivantes :

\begin{enumerate}
    \item[(a)] \( -x^2 - x - 6 \leq 0 \) 
    \item[(c)] \( 4x^2 - 4x + 1 \leq 0 \) 
\end{enumerate}


\section*{\underline{Exercice 5 :}}

\begin{enumerate}
    \item Factoriser les expressions suivantes :
    \[
    A(x) = 25x^2 - 4 + (5x + 2)(x - 2) \quad ; \quad C(x) = x^3 + 1 - 2x(x^2 - 1)
    \]
    \[
    B(x) = x^3 - 8 + (x - 2)(2x - 3)
    \]
    
    \item Résoudre les équations et les inéquations suivantes :
    \[
    |4x + 3| = 2x + 1 \quad ; \quad |-2x + 4| \leq 6 \quad ; \quad |x - 2| = |7 - 3x| \quad ; \quad |7x - 2| \geq 2.
    \]
\end{enumerate}

\section*{\underline{Exercice 6 :}}

 Résoudre dans \( \mathbb{R} \times \mathbb{R} \) les systèmes d’équations suivants :
    
        \( S_1: 
        \begin{cases} 
            3x + 2y = 7 \\
            -4x + 5y = 6
        \end{cases} \)
        \quad ; \quad
        \( S_2: 
        \begin{cases} 
            -2x - y = 5 \\
            -8x + 7y = -13
        \end{cases} \)
        \quad ; \quad
        \( S_3: 
        \begin{cases} 
            x + y = 6 \\
            -3x - 17y = -18
        \end{cases} \)

\end{document}
