\documentclass[a4paper,12pt]{article}
\usepackage{graphicx}
\usepackage[a4paper, top=0cm, bottom=2cm, left=2cm, right=2cm]{geometry} % Ajuste les marges
\usepackage{xcolor} % Pour ajouter des couleurs
\usepackage{hyperref} % Pour avoir des références colorées si nécessaire
\usepackage{eso-pic} % Pour ajouter des éléments en arrière-plan
\usepackage[french]{babel}
\usepackage[T1]{fontenc}
\usepackage{mathrsfs}
\usepackage{amsmath}
\usepackage{amsfonts}
\usepackage{amssymb}
\usepackage{tkz-tab}
\usepackage{tcolorbox} % Pour encadrer avec fond coloré
\usepackage{tikz}
\usetikzlibrary{arrows, shapes.geometric, fit}

% Définition de l'encadré adaptatif avec fond jaune
\newtcolorbox{resultbox}{
    colback=red!30, % Fond rouge clair
    colframe=black, % Bordure noire fine
    sharp corners, % Coins nets
    boxrule=0.5pt, % Contour léger
    boxsep=2pt, % Espacement interne
    left=5pt, right=5pt, top=2pt, bottom=2pt, % Marges internes
}

\begin{document}

\hrule % Barre horizontale
% En-tête
\begin{center}
    \begin{tabular}{@{} p{5cm} p{5cm} p{5cm} @{}} % 3 colonnes avec largeurs fixées
        Lycée Dindéfélo & \quad\quad\quad Test 1 & 11 Mars 2025 \\
    \end{tabular}
    \\[-0.01cm] % Ajuster l'espace vertical entre le tableau et la barre
    \hrule % Barre horizontale
\end{center}
\begin{center}
    \textbf{\Large Trinôme du second degré} \\[0.2cm]
    \textbf{\large Professeur : M. BA} \\[0.2cm]
    \textbf{Classe : 2ndL} \\[0.2cm]
    \textbf{\small Durée : 10 minutes} \\[0.2cm]
    \textbf{\small Note :\quad\quad /5}
\end{center}

% Nom de l'élève
\textbf{\small Nom de l'élève :} \underline{\hspace{8cm}} \\[0.5cm]

\( A(x) = 2x^2 - 4x - 30 \)

\begin{enumerate}
    \item Calculer le discriminant \(\Delta\).
    \item Déterminer la forme canonique du trinôme.
    \item Factoriser le trinôme, si possible.
\end{enumerate}

\section*{Correction de l'exercice}

Soit le trinôme \( A(x) = 2x^2 - 4x - 30 \).

\begin{enumerate}
    \item \textbf{Calcul du discriminant}  
    Le discriminant \(\Delta\) est donné par :
    
$
\begin{aligned}
 \Delta &= b^2 - 4ac\\ 
 &= (-4)^2 - 4 \times 2 \times (-30) \\
 &= 16 + 240\\
 &= 256
\end{aligned}
$

Ainsi, le discriminant est : 

\begin{resultbox}
    \[
    \mathbf{\Delta = 256}
    \]
\end{resultbox}

    \item \textbf{Détermination de la forme canonique}  
    
    La forme canonique du trinôme est :

$
\begin{aligned}
    A(x)&= a \left[ \left(x + \frac{b}{2a}\right)^2 - \frac{b^{2}-4ac}{4a^{2}}\right]\\
        &=  2 \left[ \left(x - \frac{4}{2 \times 2 }\right)^2 - \frac{256}{4 \times 2^2}\right]\\
        &=2 \left[\left( x - 1 \right)^2 - 16\right]
\end{aligned}
$

Ainsi, la forme canonique est :

\begin{resultbox}
    \[
    \mathbf{A(x) = 2 \left[\left( x - 1 \right)^2 - 16\right]}
    \]
\end{resultbox}

    \item \textbf{Factorisation du trinôme}  
    
    Comme \(\Delta > 0\), le trinôme admet deux racines réelles distinctes \( x_1 \) et \( x_2 \) :
    
$
\begin{aligned}
        x_1 &= \frac{-b - \sqrt{\Delta}}{2a}\\
        &= \frac{4 - \sqrt{256}}{4}\\
        &= \frac{4 - 16}{4}\\
        &= -3
\end{aligned}
$

\begin{resultbox}
    \[
    \mathbf{x_1 = -3}
    \]
\end{resultbox}

$
\begin{aligned}
        x_2 &= \frac{-b + \sqrt{\Delta}}{2a}\\
        &= \frac{4 + \sqrt{256}}{4}\\
        &= \frac{4 + 16}{4}\\
        &= 5
\end{aligned}
$

\begin{resultbox}
    \[
    \mathbf{x_2 = 5}
    \]
\end{resultbox}

    La factorisation est :

\begin{resultbox}
    \[
    \mathbf{A(x) = 2(x - 5)(x + 3)}
    \]
\end{resultbox}
    
\end{enumerate}

\end{document}
